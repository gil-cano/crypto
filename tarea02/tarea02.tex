% XeLaTeX can use any Mac OS X font. See the setromanfont command below.
% Input to XeLaTeX is full Unicode, so Unicode characters can be typed directly into the source.

% The next lines tell TeXShop to typeset with xelatex, and to open and save the source with Unicode encoding.

%!TEX TS-program = xelatex
%!TEX encoding = UTF-8 Unicode

\documentclass[12pt]{article}
\usepackage{geometry}                % See geometry.pdf to learn the layout options. There are lots.
\geometry{letterpaper}                   % ... or a4paper or a5paper or ... 
%\geometry{landscape}                % Activate for for rotated page geometry
\usepackage[parfill]{parskip}    % Activate to begin paragraphs with an empty line rather than an indent
\usepackage{graphicx}
\usepackage{amssymb}
\usepackage{scrextend}

% Will Robertson's fontspec.sty can be used to simplify font choices.
% To experiment, open /Applications/Font Book to examine the fonts provided on Mac OS X,
% and change "Hoefler Text" to any of these choices.

\usepackage{fontspec,xltxtra,xunicode}
\defaultfontfeatures{Mapping=tex-text}
\setromanfont[Mapping=tex-text]{Times}
\setsansfont[Scale=MatchLowercase,Mapping=tex-text]{Gill Sans}
\setmonofont[Scale=MatchLowercase]{Andale Mono}

\topmargin=-50pt
\textheight=630pt
\title{Criptografía y Seguridad, Tarea 2}
\author{Fecha de Entrega: Lunes 15 de abril de 2013}
\date{}                                           % Activate to display a given date or no date

\begin{document}
\maketitle

% For many users, the previous commands will be enough.
% If you want to directly input Unicode, add an Input Menu or Keyboard to the menu bar 
% using the International Panel in System Preferences.
% Unicode must be typeset using a font containing the appropriate characters.
% Remove the comment signs below for examples.

% \newfontfamily{\A}{Geeza Pro}
% \newfontfamily{\H}[Scale=0.9]{Lucida Grande}
% \newfontfamily{\J}[Scale=0.85]{Osaka}

% Here are some multilingual Unicode fonts: this is Arabic text: {\A السلام عليكم}, this is Hebrew: {\H שלום}, 
% and here's some Japanese: {\J 今日は}.

\begin{enumerate}

\item Una propiedad importante en la seguridad de DES es que las cajas-S no son lineales.
Muestra que $S_1(x_1) \oplus S_1(x_2)  \ne S_1(x_1 \oplus x_2) $, para:
\begin{enumerate}
\item $x_1 = 000000, x_2 = 000001$
\item $x_1 = 111111, x_2 = 100000$
\item $x_1 = 101010, x_2 = 010101$
\end{enumerate}

\item ¿Cuál es la salida en la primera ronda del algoritmo DES cuando el texto claro y la llave son todos 1?

\item Usando en DES una palabra de entrada que tiene un 1 en el bit 57 y en todos los demás 0, y una llave de puros ceros:
\begin{enumerate}
\item ¿Cuantas cajas-S tienen una entrada diferente comparado con el caso en que todos los bits de la palabra de entrada   son cero?
\item ¿Cuál es la salida después de la primera ronda?
\item ¿Cuantos bits de salida cambiaron después de la primera ronda,  respecto a el caso en que todos los bits de la palabra de entrada son cero?
\end{enumerate}

\item Sean $p = 41$ y $p = 17$ los dos primos dados como parámetros de RSA.
\begin{enumerate}
\item ¿Cuál de los parámetros $e_1 = 32, e_2 = 49$ es un exponente valido para RSA?
\item Calcula la llave privada correspondiente $k_{pr} = (p, q, d)$
\end{enumerate}

\item Cifra y descifra usando el algoritmo RSA con los siguientes parámetros:
\begin{enumerate}
\item $p = 3$, $q = 11$, $d =  7$, $x = 5$
\item $p = 5$, $q = 11$, $e =  3$, $x = 9$
\end{enumerate}

\item Dado un esquema de firma digital con RSA y llave publica ($n = 9797, e = 131$), ¿Cuáles de las siguientes firmas
es valida?
\begin{enumerate}
\item $(x = 123, sig(x) = 6292)$
\item $(x = 4333, sig(x) = 4768)$
\item $(x = 4333, sig(x) = 1424)$
\end{enumerate}


\end{enumerate}
\end{document}  