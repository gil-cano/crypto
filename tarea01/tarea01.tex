% XeLaTeX can use any Mac OS X font. See the setromanfont command below.
% Input to XeLaTeX is full Unicode, so Unicode characters can be typed directly into the source.

% The next lines tell TeXShop to typeset with xelatex, and to open and save the source with Unicode encoding.

%!TEX TS-program = xelatex
%!TEX encoding = UTF-8 Unicode

\documentclass[12pt]{article}
\usepackage{geometry}                % See geometry.pdf to learn the layout options. There are lots.
\geometry{letterpaper}                   % ... or a4paper or a5paper or ... 
%\geometry{landscape}                % Activate for for rotated page geometry
\usepackage[parfill]{parskip}    % Activate to begin paragraphs with an empty line rather than an indent
\usepackage{graphicx}
\usepackage{amssymb}
\usepackage{scrextend}

% Will Robertson's fontspec.sty can be used to simplify font choices.
% To experiment, open /Applications/Font Book to examine the fonts provided on Mac OS X,
% and change "Hoefler Text" to any of these choices.

\usepackage{fontspec,xltxtra,xunicode}
\defaultfontfeatures{Mapping=tex-text}
\setromanfont[Mapping=tex-text]{Times}
\setsansfont[Scale=MatchLowercase,Mapping=tex-text]{Gill Sans}
\setmonofont[Scale=MatchLowercase]{Andale Mono}

\topmargin=-50pt
\textheight=630pt
\title{Criptografía y Seguridad, Tarea 1}
\author{Fecha de Entrega: Lunes 18 de febrero de 2013}
\date{}                                           % Activate to display a given date or no date

\begin{document}
\maketitle

% For many users, the previous commands will be enough.
% If you want to directly input Unicode, add an Input Menu or Keyboard to the menu bar 
% using the International Panel in System Preferences.
% Unicode must be typeset using a font containing the appropriate characters.
% Remove the comment signs below for examples.

% \newfontfamily{\A}{Geeza Pro}
% \newfontfamily{\H}[Scale=0.9]{Lucida Grande}
% \newfontfamily{\J}[Scale=0.85]{Osaka}

% Here are some multilingual Unicode fonts: this is Arabic text: {\A السلام عليكم}, this is Hebrew: {\H שלום}, 
% and here's some Japanese: {\J 今日は}.

\begin{enumerate}
\item ¿Cuántas transformaciones afines distintas hay para el alfabeto español de 27 letras? Justifique su respuesta.

\item ¿Cuántas transformaciones afines distintas hay para el alfabeto griego de 24 letras? Justifique su respuesta.

\item Utilice la transformación afin $3x + 5\ (mod\ 26)$ para cifrar el siguiente texto. Ignore los acentos y signos de puntuación:

\begin{addmargin}[1em]{2em} % 1em left, 2em right
Siempre tengo un plan de emergencia cuando  tomo algún riesgo: si todo lo demás falla, voy a morir una muerte horrible
y dolorosa. En efecto, no es un buen plan de emergencia, pero
es un plan.

(Walter Slovotsky, en {\it El camino a casa} de Joel Rosenberg)
\end{addmargin}

\item El siguiente texto fue cifrado con una transformación afín sobre el alfabeto español de 27 letras.
La primer palabra es "cuando". Diga cuál fue la transformación de cifrado, y decifre el mensaje. (Sugerencia: encuentre la función inversa, que también es una transformación afín.)

\begin{addmargin}[1em]{2em} % 1em left, 2em right
\setromanfont[Mapping=tex-text]{Courier}
JEBZN HAQZY BCGKH WIHTB CBJQS AHNHR HLHWG FRQLB SBLEQ
RQBZV EJCHL HWQKQ VLRHL QYBSR QWWGZ HRHCB JQZNH EYRBW
BNBVW
\end{addmargin}

\setromanfont[Mapping=tex-text]{Times}

\item Decifre el siguiente texto, que utiliza el alfabeto estándar. De ser posible, describa la construcción del alfabeto de cifrado:

\begin{addmargin}[1em]{2em} % 1em left, 2em right
\setromanfont[Mapping=tex-text]{Courier}
PADTYR AT PZDGODVDT YEJCMOD OEVGDJ L YMDQGDHDJ VDT BAETRJ
PRHR ORJ YE TEWVRT IDGD UAMETEJ TR JDBET TDYD DOQATDJ
IEGJRTDJ JE IGEQATVDT IRG UAE OD PMETPMD TR JE OEKDTVD
L OR VGMVAGD EJ IRG EJVR UAMETEJ IAEYET ETVETYEG OD
GECAVDPMRT LD JDBET UAE TR EJ TEPEJDGMD L IAEYET JM
DJM OR YEJEDT YEVEPVDG DO PZDGODVDT IRG JM HMJHR ORJ
UAE TR IAEYET TR JDBGMDT OD YMCEGETPMD ETVGE OD KEGYDYEGD
GEJIAEJVD L ORJ TAEKRJ YMDQGDHDJ L OEVGDJ UAE EO PZDGODVDT
IGRYAPMGMD YEPODGDTYR UAE ZD YEGGRVDYR D ORJ CMORJRCRJ
DAQAJVR YE HRGQDT ATD PROEPPMRT YE IDGDYRFDJ
\end{addmargin}

\setromanfont[Mapping=tex-text]{Times}

\item Decifre el siguiente texto, que utiliza el alfabeto estándar. De ser posible, describa la construcción del alfabeto de cifrado: 

\begin{addmargin}[1em]{2em} % 1em left, 2em right
\setromanfont[Mapping=tex-text]{Courier}
BLKLI VOXLM GIZIR LHLBF MKILW FXGLZ XZYZW LZYHL IYLVM
VITRZ VOVXG IRXZW RIVXG ZNVMG VBOZF GRORA LXLMX ZHRFM
XRVMG LKLIX RVMGL WVVUR XRVMX RZVHG LBXLN KFVHG LWVUF
VIGVN VGZOV HGLBX LMHXR VMGVX LMHGZ MGVNV MGVBK FVWLH
LKLIG ZIUZX RONVM GVOLH NZHVC GIVNZ WLHXZ NYRLH ZNYRV
MGZOV HVHGL HHLMS VXSLH JFVKZ IGRVM WLWVO ZRIIV UFGZY
OVKIL KLHRX RLMWV JFVMR MTFMH VIKFV WVXIV ZIFMH VINHK
VIUVX GLJFV VOIVW FXVEF VHGIZ GLMGZ GVLIZ ZOZMZ WZ\hspace*{21pt}
\end{addmargin}

\setromanfont[Mapping=tex-text]{Times}

\end{enumerate}

\end{document}  