\documentclass[11pt, oneside]{article}   	% use "amsart" instead of "article" for AMSLaTeX format
\usepackage{geometry}                		% See geometry.pdf to learn the layout options. There are lots.
\geometry{letterpaper}                   		% ... or a4paper or a5paper or ... 
%\geometry{landscape}                		% Activate for for rotated page geometry
\usepackage[parfill]{parskip}    		% Activate to begin paragraphs with an empty line rather than an indent
\usepackage{graphicx}				% Use pdf, png, jpg, or eps� with pdflatex; use eps in DVI mode
								% TeX will automatically convert eps --> pdf in pdflatex		
\usepackage{amssymb}

\title{Brief Article}
\author{The Author}
%\date{}							% Activate to display a given date or no date

\begin{document}
\maketitle
%\section{}
%\subsection{}

When you have some material, hit the �Typeset� button at the top left of the window. A dialog will appear asking you to save the document. When TEX typesets, it creates three or four additional files, so it is not a good idea to save directly to a location with many other files. Instead, put your source in a folder within such a location. To do that, navigate to a reasonable location, say ~/Documents, and then hit the �New Folder� button at the bottom of the dialog. Accept the default name or choose another and create the folder. Then name the document and save it. The default �Untitled� name will do

As soon as you save, TEX will typeset the document and open a second window showing the result.

Go back to the original window and add some additional text. Hit �Typeset� again. This time TEX immediately typesets the new material and updates the output window.

Now quit the program.

\end{document}  